\documentclass[12pt]{article}
\usepackage[T1]{fontenc}
\usepackage[latin2]{inputenc}
\usepackage[english]{babel}
\usepackage{palatino}
\usepackage{amsmath}   
\usepackage{amssymb}
\usepackage{mathtools}
\setlength{\parindent}{0pt}
\usepackage[left=2.5cm,right=2.5cm,top=4cm,bottom=4cm]{geometry}

\begin{document}
\author{Zsolt Borsi}

\begin{center}
{\Large Homework 2}
\end{center}
\bigskip

Let $A=\left[1..6\right]$ be a statespace, $S\subseteq A \times (\bar{A}\cup\{fail\})^{**}$ a program over the statespace $A$.
\[
S = \begin{Bmatrix*}[l]
1\rightarrow<1,2,5,1> & 2\rightarrow<2,4,3,5,6> & 2\rightarrow<2,2,2,\ldots>\\
3\rightarrow<3,1> & 3\rightarrow<3,2,4> & 3\rightarrow<3,5,2,4,1>\\
4\rightarrow<4,1,fail> & 5\rightarrow<5,3,2,4> & 5\rightarrow<5,3,6,1>\\
6\rightarrow<6,1,4> & 6\rightarrow<6,1,fail> & \\
\end{Bmatrix*}
\]

Let $F\subseteq A \times A$ denote the following problem: $F=\left\{\,(1,1),(3,4),(3,1),(5,1),(5,2),(5,4)\,\right\}$\\
\bigskip

\textbf{Question}\\ 
Let $S_1$ and $S_2$ be programs, let $F_1$ and $F_2$ be problems over the same statespace.\\
Statement1: if $S_1$ solves $F_1$ and $F_1 \subseteq F_2$ then $S_1$ solves $F_2$.\\
Statement2: if $S_1$ solves $F_1$ and $F_2 \subseteq F_1$ then $S_1$ solves $F_2$.\\
Statement3: if $S_1$ solves $F_1$ and $S_1 \subseteq S_2$ then $S_2$ solves $F_1$.\\
Statement4: if $S_1$ solves $F_1$ and $S_2 \subseteq S_1$ then $S_2$ solves $F_1$.\\

Informally, for example, Statement4 says that if a program $S_1$ solves a given problem $F_1$ then a smaller program $S_2$ also solves the same problem $F_1$. Interestingly, only this fourth statement is true out of the four statements.\\

Your task is to find a counter-example for Statement3.\\
More precisely: consider the program $S$ and problem $F$ given above in this exercise. Find a program $S_2$ such that $S \subseteq S_2$ and $S_2$ does not solve $F$. Explain why your $S_2$ program does not solve $F$.

\rule{\textwidth}{0.4pt}
\textbf{Recall}\\
$S$ program solves problem $F$ if
\begin{enumerate}
	\item $\mathcal{D}_F \subseteq \mathcal{D}_{P[S]}$
	\item $\forall a \in \mathcal{D}_F\colon P[S](a) \subseteq F(a)$
\end{enumerate}  		
\end{document}